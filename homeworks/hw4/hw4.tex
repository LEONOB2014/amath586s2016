\documentclass[10pt]{article}

\usepackage{graphicx}
\usepackage{amsmath,amsfonts,amssymb}

% use different colors for links:
\usepackage{color}
\definecolor{darkgreen}{rgb}{0.1,0.5,0.1}
\definecolor{darkblue}{rgb}{0.2,0.2,1.0}
\usepackage[colorlinks=true,linkcolor=darkblue,citecolor=darkblue,
            filecolor=darkblue,urlcolor=darkgreen]{hyperref}


\setlength{\textwidth}{6.2in}
\setlength{\oddsidemargin}{0.3in}
\setlength{\evensidemargin}{0in}
\setlength{\textheight}{8.9in}
\setlength{\voffset}{-1in}
\setlength{\headsep}{26pt}
\setlength{\parindent}{0pt}
\setlength{\parskip}{5pt}




% a few handy macros

\newcommand\matlab{{\sc matlab}}
\newcommand{\goto}{\rightarrow}
\newcommand{\bigo}{{\mathcal O}}
\newcommand{\half}{\frac{1}{2}}
%\newcommand\implies{\quad\Longrightarrow\quad}
\newcommand\reals{{{\rm l} \kern -.15em {\rm R} }}
\newcommand\complex{{\raisebox{.043ex}{\rule{0.07em}{1.56ex}} \hskip -.35em {\rm C}}}


% macros for matrices/vectors:

% matrix environment for vectors or matrices where elements are centered
\newenvironment{mat}{\left[\begin{array}{ccccccccccccccc}}{\end{array}\right]}
\newcommand\bcm{\begin{mat}}
\newcommand\ecm{\end{mat}}

% matrix environment for vectors or matrices where elements are right justifvied
\newenvironment{rmat}{\left[\begin{array}{rrrrrrrrrrrrr}}{\end{array}\right]}
\newcommand\brm{\begin{rmat}}
\newcommand\erm{\end{rmat}}

% for left brace and a set of choices
\newenvironment{choices}{\left\{ \begin{array}{ll}}{\end{array}\right.}
\newcommand\when{&\text{if~}}
\newcommand\otherwise{&\text{otherwise}}
% sample usage:
%  \delta_{ij} = \begin{choices} 1 \when i=j, \\ 0 \otherwise \end{choices}


% for labeling and referencing equations:
\newcommand{\eql}{\begin{equation}\label}
\newcommand{\eqn}[1]{(\ref{#1})}
% can then do
%  \eql{eqnlabel}
%  ...
%  \end{equation}
% and refer to it as equation \eqn{eqnlabel}.  


% some useful macros for finite difference methods:
\newcommand\unp{U^{n+1}}
\newcommand\unm{U^{n-1}}

% for chemical reactions:
\newcommand{\react}[1]{\stackrel{K_{#1}}{\rightarrow}}
\newcommand{\reactb}[2]{\stackrel{K_{#1}}{~\stackrel{\rightleftharpoons}
   {\scriptstyle K_{#2}}}~}

% Parts:

% set enumerate to give parts a, b, c, ...  rather than numbers 1, 2, 3...
\renewcommand{\theenumi}{\alph{enumi}}
\renewcommand{\labelenumi}{(\theenumi)}

% set second level enumerate to give parts i, ii, iii, iv, etc.
\renewcommand{\theenumii}{\roman{enumii}}
\renewcommand{\labelenumii}{(\theenumii)}

  % input some useful macros

\begin{document}

% header:
\hfill\vbox{\hbox{AMath 586 / ATM 581}
\hbox{Homework \#4}\hbox{Due Tuesday, May 10, 2016}}

{\bf Name:} Your name here
\vskip 5pt

Homework is due to Canvas by 11:00pm PDT on the due date.

To submit, see \url{https://canvas.uw.edu/courses/962872/assignments/3270587}


%--------------------------------------------------------------------------
\vskip 1cm
\hrule
{\bf Problem 1}  


Consider the following method for solving the heat equation
$u_t=u_{xx}$:
\[
U_i^{n+2} = U_i^n + \frac{2k}{h^2}(U_{i-1}^{n+1} - 2U_i^{n+1} +
U_{i+1}^{n+1}).
\]
\begin{enumerate}
\item Determine the formal order of accuracy of this method 
(in both space and time) based on computing the local truncation error.

\item Suppose we take $k=\alpha h^2$ for some fixed $\alpha>0$ and refine
the grid.  Show that this method fails to be 
Lax-Richtmyer stable for any choice of $\alpha$.

Do this in two ways: 
\begin{itemize}
\item Consider the MOL interpretation and the stability region of
the time-discretization being used.
\item Use von Neumann analysis and solve a quadratic equation for $g(\xi)$.
\end{itemize} 

\item What if we take $k=\alpha h^3$ for some fixed $\alpha>0$ and refine
the grid. Would this method be convergent?

\end{enumerate}


% uncomment the next two lines if you want to insert solution...
%\vskip 1cm
%{\bf Solution:}

% insert your solution here!

%--------------------------------------------------------------------------
\vskip 1cm
\hrule
{\bf Problem 2}  


\begin{enumerate} 
\item Suppose we want to approximate the fourth derivative $u_{xxxx}$
numerically. One approach is to apply the second derivative operator twice,
i.e., if 
\[
D_2 U_i = \frac 1 {h^2}(U_{i-1} - 2U_i + U_{i+1})
\]
then use
\[
u_{xxxx}(x_i) \approx \frac 1 {h^2}(D_2 U_{i-1} - 2D_2 U_i + D_2 U_{i+1}).
\]
When you write these terms out and combine them, this gives a finite
difference approximation that is a linear combination of five values
$U_{i-2},~U_{i-1},~U_i,~U_{i+1},~U_{i+2}$.  
Write out this approximation.


\item The notebook \verb+notebooks/vxx.ipynb+ from the
class repository illustrates how to set up a matrix $A$ corresponding
to the centered second difference operator, both for Dirichlet
boundary conditions and for periodic boundary conditions.  Adapt
the {\it periodic case} to test out this approximation to $u_{xxxx}$.
Note that the matrix will now be pentadiagonal (5 nonzero diagonals)
and will also have more terms in the corners that arise from periodicity.
Test the accuracy on the periodic function $v(x) = \sin^2(2\pi x)$
for which it's not too bad to compute the exact fourth derivative.  
You should observe second order accuracy.

\item Determine analytically the eigenvalues of this matrix.  Recall that
since it is a circulant matrix, the $(m+1)\times (m+1)$ version of this
matrix with $h=1/(m+1)$  will have eigenvectors $u^p$ with components 
$u^p_j = e^{2\pi i pjh}$.  Note that since the matrix is symmetric it should
have real eigenvalues, so simplify the eigenvalue expressions to make this
clear. 

\item Check that for fixed $p$ the eigenvalue $\lambda_p$ of the matrix
agrees to $\bigo(h^2)$  with the eigenvalue $(2\pi p)^4$ of the differential
operator $\partial_x^4$ on the interval $[0,1]$ with periodic boundary
conditions.

\item Suppose we want to solve the equation $u_t = -\kappa u_{xxxx}$
for some $\kappa >0$ and we use the difference approximation derived
above for the spatial term but then apply Forward Euler in time to
the resulting MOL system.  What is the stability restriction relating
$k$ and $h$ that would have to be satisfied for the method to be
convergent?  

\end{enumerate}


% uncomment the next two lines if you want to insert solution...
%\vskip 1cm
%{\bf Solution:}

% insert your solution here!

%--------------------------------------------------------------------------

\vskip 1cm
\hrule
{\bf Problem 4}  


\begin{enumerate} 
\item The notebook \verb+notebooks/HeatEquation_CN.ipynb+ from the
class repository solves the heat equation $u_t = \kappa u_{xx}$
(with $\kappa = 0.02$) using the Crank-Nicolson method.

Modify the notebook to also 
produce a log-log plot of the error versus $h$ with the timestep $k$ chosen
to be $k=2h$ (this is slightly different than the test currently performed in
the notebook).

\item Modify this notebook to produce a new version
\verb+HeatEquation_TRBDF2.ipynb+
implements the TR-BDF2 method on the same problem.  Test it to confirm that
it is also second order accurate.  Explain how you determined the proper
boundary conditions in each stage of this Runge-Kutta method.

\item Modify the code to produce a new notebook \verb+HeatEquation_FE.ipynb+
that implements the forward Euler explicit method on the same
problem.  Test it to confirm that it is $\bigo(h^2)$ accurate as
$h\goto 0$ provided when $k = 24 h^2$ is used, which is within the
stability limit for $\kappa = 0.02$.  Note how many more time steps
are required than with Crank-Nicolson or TR-BDF2, especially on
finer grids.

\item Test the forward Euler method with $k = 26 h^2$, for which it should be
unstable.  Note that the instability does not become apparent until about
time 4.5 for the parameter values $\kappa = 0.02,~ m=39,~\beta = 150$.
Explain why the instability takes several hundred time steps to appear, and
why it appears as a sawtooth oscillation. 

{\bf Hint:} What wave numbers $\xi$ are growing exponentially for these
parameter values?  What is the initial magnitude of the most unstable
eigenmode in the given initial data?  The expression (E.30) in the book
for the Fourier transform of a Gaussian may be useful.

\end{enumerate}



% uncomment the next two lines if you want to insert solution...
%\vskip 1cm
%{\bf Solution:}

% insert your solution here!


%--------------------------------------------------------------------------

\end{document}

