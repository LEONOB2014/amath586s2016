\documentclass[10pt]{article}

\usepackage{graphicx}
\usepackage{amsmath,amsfonts,amssymb}

\usepackage{hyperref}  % for urls and hyperlinks


\setlength{\textwidth}{6.2in}
\setlength{\oddsidemargin}{0.3in}
\setlength{\evensidemargin}{0in}
\setlength{\textheight}{8.9in}
\setlength{\voffset}{-1in}
\setlength{\headsep}{26pt}
\setlength{\parindent}{0pt}
\setlength{\parskip}{5pt}




% a few handy macros

\newcommand\matlab{{\sc matlab}}
\newcommand{\goto}{\rightarrow}
\newcommand{\bigo}{{\mathcal O}}
\newcommand{\half}{\frac{1}{2}}
%\newcommand\implies{\quad\Longrightarrow\quad}
\newcommand\reals{{{\rm l} \kern -.15em {\rm R} }}
\newcommand\complex{{\raisebox{.043ex}{\rule{0.07em}{1.56ex}} \hskip -.35em {\rm C}}}


% macros for matrices/vectors:

% matrix environment for vectors or matrices where elements are centered
\newenvironment{mat}{\left[\begin{array}{ccccccccccccccc}}{\end{array}\right]}
\newcommand\bcm{\begin{mat}}
\newcommand\ecm{\end{mat}}

% matrix environment for vectors or matrices where elements are right justifvied
\newenvironment{rmat}{\left[\begin{array}{rrrrrrrrrrrrr}}{\end{array}\right]}
\newcommand\brm{\begin{rmat}}
\newcommand\erm{\end{rmat}}

% for left brace and a set of choices
\newenvironment{choices}{\left\{ \begin{array}{ll}}{\end{array}\right.}
\newcommand\when{&\text{if~}}
\newcommand\otherwise{&\text{otherwise}}
% sample usage:
%  \delta_{ij} = \begin{choices} 1 \when i=j, \\ 0 \otherwise \end{choices}


% for labeling and referencing equations:
\newcommand{\eql}{\begin{equation}\label}
\newcommand{\eqn}[1]{(\ref{#1})}
% can then do
%  \eql{eqnlabel}
%  ...
%  \end{equation}
% and refer to it as equation \eqn{eqnlabel}.  


% some useful macros for finite difference methods:
\newcommand\unp{U^{n+1}}
\newcommand\unm{U^{n-1}}

% for chemical reactions:
\newcommand{\react}[1]{\stackrel{K_{#1}}{\rightarrow}}
\newcommand{\reactb}[2]{\stackrel{K_{#1}}{~\stackrel{\rightleftharpoons}
   {\scriptstyle K_{#2}}}~}

% Parts:

% set enumerate to give parts a, b, c, ...  rather than numbers 1, 2, 3...
\renewcommand{\theenumi}{\alph{enumi}}
\renewcommand{\labelenumi}{(\theenumi)}

% set second level enumerate to give parts i, ii, iii, iv, etc.
\renewcommand{\theenumii}{\roman{enumii}}
\renewcommand{\labelenumii}{(\theenumii)}

  % input some useful macros

\begin{document}

% header:
\hfill\vbox{\hbox{AMath 586 / ATM 581}
\hbox{Homework \#2}\hbox{Due Thursday, April 14, 2016}}

\vskip 5pt

Homework is due to Canvas by 11:00pm PDT on the due date.

To submit, see
\url{https://canvas.uw.edu/courses/1038268/assignments/3254815}


%--------------------------------------------------------------------------

%--------------------------------------------------------------------------
\vskip 1cm
\hrule
{\bf Problem 1}

Which of the following Linear Multistep Methods are convergent?  For 
the ones that are not, are they inconsistent, or not zero-stable, or both?
 \begin{enumerate}
 \item $U^{n+3} = U^{n+1} + 2kf(U^n)$,
 \item $U^{n+2} = \half U^{n+1} + \half U^{n} + 2kf(U^{n+1})$,
 \item $\unp = U^n$, 
 \item $U^{n+4} = U^{n} + \frac 4 3 k(f(U^{n+3})+f(U^{n+2})+f(U^{n+1}))$,
 \item $U^{n+3} = -U^{n+2} + U^{n+1} +U^{n}+2k(f(U^{n+2})+f(U^{n+1}))$.
 \end{enumerate}



% uncomment the next two lines if you want to insert solution...
%\vskip 1cm
%{\bf Solution:}

% insert your solution here!


%--------------------------------------------------------------------------
\vskip 1cm
\hrule
{\bf Problem 2}


Consider the IVP
\begin{equation*}
\begin{split}
u_1' &= 2u_1,\\
u_2' &= 3u_1 + 2u_2,
\end{split}
\end{equation*}
with initial conditions specified at time $t=0$.  Solve this problem in two
different ways:

\begin{enumerate}
\item Solve the first equation, which only involves $u_1$, and then insert
this function into the second equation to obtain a nonhomogeneous linear
equation for $u_2$.  Solve this using (5.8).

\item Write the system as $u' = Au$ and compute the matrix exponential using
(D.35) to obtain the solution.  (See Appendix C.3 for a discussion of the
Jordan Canonical form in the defective case.)
\end{enumerate}


% uncomment the next two lines if you want to insert solution...
%\vskip 1cm
%{\bf Solution:}

% insert your solution here!


%--------------------------------------------------------------------------
\vskip 1cm
\hrule
{\bf Problem 3}

\begin{enumerate}
\item Determine the general solution to the linear difference equation
$U^{n+2} = U^{n+1} + U^n$.

\item Determine the solution to this difference equation with the starting
values $U^0=1$, $U^1=1$.  Use this to determine $U^{30}$?  
(Note, these are the {\em Fibonacci numbers}, which of course should all be
integers.)

\item Show that for large $n$ the ratio of successive Fibonacci numbers
$U^n/U^{n-1}$ approaches the ``golden ratio'' $\phi \approx 1.618034$.
\end{enumerate} 

% uncomment the next two lines if you want to insert solution...
%\vskip 1cm
%{\bf Solution:}

% insert your solution here!



%--------------------------------------------------------------------------
\newpage
%\vskip 1cm
\hrule
{\bf Problem 4}


Perform numerical experiments to confirm the claims made in Example 7.11 on
page 160, by comparing the performance of Forward Euler and the Midpoint
methods for different choices of parameters.  In particular, you might want
to try $a=1$ and compare $b=0$ with $b=1$ over $0\leq t \leq 10$, but feel
free to experiment and perhaps find other values that illustrate this better.
Explore how each method behaves for different grid resolutions in each case.

Compute the exact solution to this system (e.g. using the matrix
exponential) so that you can plot the true solution along with the computed
solution for various cases.

% uncomment the next two lines if you want to insert solution...
%\vskip 1cm
%{\bf Solution:}

% insert your solution here!


%--------------------------------------------------------------------------
\vskip 1cm
\hrule
{\bf Problem 5}

Implement the 3rd order 3-step Adams-Bashforth method for the linearized
pendulum problem from Problem 4.  As starting values, try the
following two choices:

\begin{enumerate} 
\item Use Forward Euler to compute $U^1$ and then Midpoint to compute $U^2$
from $U^0$ and $U^1$.

\item Use the 2-stage Runge-Kutta method (5.32) to compute $U^1$ and also to
compute $U^2$.
\end{enumerate}

In each case, what global order of accuracy do you observe at some fixed
time? (E.g. $t=10$ with parameters $a=1,~b=0.2$)?  Estimate this by solving
the problem with several different grid resolutions.

In each case, explain why the global order you observe agrees with what
should be expected.

% uncomment the next two lines if you want to insert solution...
%\vskip 1cm
%{\bf Solution:}

% insert your solution here!


\end{document}

